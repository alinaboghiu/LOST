\documentclass{report}
\usepackage{graphicx}
\begin{document}

%-------------------------------------------------------------------------------
%    TITLE PAGE
%-------------------------------------------------------------------------------

\begin{titlepage}
\newcommand{\HRule}{\rule{\linewidth}{0.5mm}}
\center
\textsc{\LARGE Imperial College London}  \\[0.5cm]
\textsc{\Large Department of Computing}  \\[0.5cm]
\textsc{\large Third Year Individual Project} \\[1.5cm]
\HRule \\[0.3cm]
{\huge \bfseries LOST: \\[0.2cm] The Logic Semantics Tutor} \\[0.3cm]
\HRule \\[1.5cm]

% author and supervisors
\begin{minipage}{0.4\textwidth}
\begin{flushleft} \large \emph{Author:} \\
 Alina  \textsc{Boghiu}
\end{flushleft}
\end{minipage}~
\begin{minipage}{0.4\textwidth}
\begin{flushright} \large \emph{Supervisor:} \\
 Ian \textsc{Hodkinson}
\end{flushright}
\begin{flushright} \large \emph{Second marker:} \\
 Fariba \textsc{Sadri}
\end{flushright}
\end{minipage}\\[4cm]
{\large \today}\\[3cm]
\vfill
\end{titlepage}

%-------------------------------------------------------------------------------
%    ABSTRACT
%-------------------------------------------------------------------------------

\begin{abstract}
The aim of this project was to develop a software tool that can teach the 
semantics of first order predicate logic to students by helping them visualise 
the process of sentence evaluation. Thus, the focus was on developing an 
intuitive and engaging user interface to show and allow modification of 
structures, signatures and sentences, as well as provide relevant exercises for 
the student to practice with. The latter is arguably the most important feature 
of this tool and an addition to the functionality of the previous LOST. The user
can now ask to solve 3 automatically generated types of puzzles, within a 
tutorial environment that measures their progress. Completing each lesson is an 
actual achievement and provides real confirmation of understanding the semantics
of first order logic. \\ \\
For robustness, the tool is linked to a lesson database which can be accessed by
students and lecturers alike. The former also have permissions to add, edit or
remove lessons and see their student's progress. \\ \\
I believe these are firm grounds for many possible extensions (such as a
Hintikka game) and can be of real use, standalone or alongside the first year
predicate logic course. This report will provide further detail of its 
implementation and purpose.
\end{abstract}

%-------------------------------------------------------------------------------
%    ACKNOWLEDGEMENTS
%-------------------------------------------------------------------------------

\subsection*{\centering Acknowledgements}
I would like to express gratitude and appreciation to my supervisor, \emph{Ian 
Hodkinson}, for his continuous support and motivation that have guided me 
towards completing this project, as well as to my second marker, \emph{Fariba
Sadri}, for her useful remarks and suggestions.\\ \\
Altogether, the resources, staff and students of the Department of Computing 
have made working on this project enjoyable every day.

%-------------------------------------------------------------------------------
    \tableofcontents
%-------------------------------------------------------------------------------

%-------------------------------------------------------------------------------
%   INTRODUCTION
%-------------------------------------------------------------------------------

\chapter{Introduction} % 3-4 pages, solid text
 % what is the project about
 % what did I set out to achieve, relevance, importance, difficulty
 % key aspects, non-technical
 % assume reader knows the basic logic operators and in general prop logic

%-------------------------------------------------------------------------------
%   BACKGROUND
%-------------------------------------------------------------------------------

\chapter{Background} % 10-20 pages
\section{First order predicate logic semantics}
In order to understand the product I am aiming for we should first take a look 
at what first order predicate logic is and why its semantics can be tricky. As 
an extension of propositional logic, it expresses statements such as \emph{
Socrates is a man} in much more detail. While propositional logic would regard
this sentence as atomic and simply assign it a truth value, predicate logic
provides a way of describing its internal structure and of evaluating it inside
a relevant context. It does this by introducing:

	\begin{itemize}
	\item \emph{Constants} 
  - which name the objects (or terms) inside a context (e.g.\ Socrates)
	\item \emph{Relations}
  - which describe properties of the objects they take as arguments or, in the
    case of nullary relations (which take no arguments), general properties of
    the structure (e.g.\ \emph{man} is a unary relation that describes Socrates)
	\end{itemize}

\noindent These elements form the signature of a structure (i.e.\ the technical 
term for the context we need in order to evaluate a logic formula and assign it
a truth value). For a computer scientist it may be easier to understand that the
structure contains instances of the signature's elements, as well as other
objects, which one can iterate over with variables. Using these concepts, we can
now rewrite the sentence as \emph{man(Socrates)}. Two questions arise: First,
which is the object that Socrates describes. Second, what does it mean for
something to be a man. If we take our structure to be an imaginary world of
dwarfs and name one of them Socrates, our sentence would be false. However in
the context of the real world where everyone is human and Socrates names the
famous philosopher, the sentence is true. \\ \\
Next, if we want to express \emph{All men are mortal} we must introduce the two
quantifiers. This was not possible in first order propositional logic and
understanding quantifiers and how the iterate over the objects in a structure is
the main source of distress amongst students. The quantifiers are:

	\begin{itemize}
	\item \emph{$\exists$ (Exists)} 
  - which checks that there is at least one object in the structure that
    satisfies the sentence it refers to and makes it true
	\item \emph{$\forall$ (For all)}
  - which checks that all of the objects in the structure satisfy the sentence
    it refers to and each make it true.
	\end{itemize}

\noindent Now the sentence can now be written as
\emph{$\forall$(men(x) $\rightarrow$ mortal(x))}
and we refer to x as a variable which in this case is bound by the for all
quantifier. Another aspect that the user must understand is that sentences
containing unbound variables are valid but cannot be evaluated to a truth
value.\\ \\
When considering my own aproach, I decided a good product would offer a way of
visualising all of this and allow the user to play around with structures and
sentences, as experimenting would make it easier to understand the inner works.

\section{Existing solutions}
  % LOST
  % other software and ideas and their limitations
  % My chosen approach: a tutorial

%-------------------------------------------------------------------------------
%   MAIN BODY
%-------------------------------------------------------------------------------

\chapter{Approach and Implementation details}

\section{The Evaluator}
The structure of my project's back end naturally lays upon the 3 main parts of
logic semantics: structures, signatures and sentences. Together these form the
Evaluator, a package which makes them interact correctly. With no front end, it
can still take a sentence input through the entire journey to a boolean outcome.
In fact this is the journey we will make to better understand its components.
\section{The User Interface}
\section{The Lesson Generator}
\section{The Lesson Database}

%-------------------------------------------------------------------------------
%   EVALUATION
%-------------------------------------------------------------------------------

\chapter{Evaluation}
\section{Quantitative evaluation}
 % minimum viable product, robustness, bugs!, 
\section{Qualitative evaluation}
 % have I done it well, is it appealing, performant, pretty

%-------------------------------------------------------------------------------
%   CONCLUSIONS AND FUTURE WORK
%-------------------------------------------------------------------------------

\chapter{Conclusions and Future Work}
\section{Learning outcomes}
\section{Potential improvements}
\section{Potential extensions}

%-------------------------------------------------------------------------------
%   BIBIOGRAPHY
%-------------------------------------------------------------------------------

\begin{thebibliography}{9}
%\bibitem{lamport94}
%  Leslie Lamport,
%  \emph{\LaTeX: A Document Preparation System}.
%  Addison Wesley, Massachusetts,
%  2nd Edition,
%  1994.
\end{thebibliography}

%-------------------------------------------------------------------------------
%   APPENDIX
%-------------------------------------------------------------------------------

\appendix
\chapter{Short demo}
\chapter{Program Listings}

%-------------------------------------------------------------------------------

\end{document}

