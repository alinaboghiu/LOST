\documentclass{article}
\usepackage{fullpage}

\begin{document}

%-------------------------------------------------------------------------------
%    TITLE PAGE
%-------------------------------------------------------------------------------

\begin{titlepage}
\newcommand{\HRule}{\rule{\linewidth}{0.5mm}}
\center
\textsc{\LARGE Imperial College London}  \\[1.5cm]
\textsc{\Large Department of Computing}  \\[0.5cm]
\textsc{\large Third Year Individual Project} \\[0.5cm]

\HRule \\[0.6cm]
{\huge \bfseries LOST: The LOgic Semantics Tutor} \\[0.3cm]
\HRule \\[1.5cm]

\begin{minipage}{0.4\textwidth}

% author
\begin{flushleft} \large \emph{Author:} \\
Alina  \textsc{Boghiu}
\end{flushleft}

% supervisor
\end{minipage}~
\begin{minipage}{0.4\textwidth}
\begin{flushright} \large \emph{Supervisor:} \\
Ian \textsc{Hodkinson}
\end{flushright}

\end{minipage}\\[4cm]

{\large \today}\\[3cm]
\vfill
\end{titlepage}

%-------------------------------------------------------------------------------

\section{Introduction}		% 1-3 pages
First order logic is a powerful tool, of great importance in computing and mathematics. For students it is a way of practicing and developing important logical skills and it provides a formal way of studying and understanding common mathematical structures. However it is difficult for them to learn the semantics of first order logic and with this being the key to assigning a truth value to any first-order logical sentence, the issue is quite severe.\\

On the current market, there are a considerable number of tools designed to help teach natural deduction or equivalances,but very few attempts have been made to design something that will acompany the student in learning the semantics of first order logic. Feedback is currently restricted to lectures and tutorials which means this kind of instant guidance from a piece of computer software would make a great impact in the way students learn, improving both their and the teacher's experience.\\

Finally, although the idea sounds simple it is not easy to implement in an attractive way to a user. The risk of making yet another slightly interactive tutorial is quite high therefore my aproach was to make sure this tool engages the student fully, whilst somehow motivating them to try and practice with it as much as they would with any computer game. This shaped the main idea for tackling the problem into designing a computer game where the user is assigned an avatar of his choice which they then have to train into mastering first order logic semantics.

minimum viable product features

	% What is the problem?
	% Why is it interesting?
	% What’s your main idea for solving it?

\section{Background}		% 10-20 pages
	% Describe as many approaches as possible
	% Desribe previous work and solutions
	% Point out the ideas i'll use in my own work

\section{Project Plan}		% 1-2 pages
	% what needs to be done
	% timetable: key milestones and backup 

\section{Evaluation}		% 1-2 pages
	% What makes it successful?
	% What extestions has it brought?
	% Quality?

\end{document}
